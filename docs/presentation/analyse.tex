\begin{frame}
  \frametitle{Darstellung des Model-View-Controller- Konzeptes}
  \begin{itemize}
    \item Welche Art der Applikation?
    \begin{itemize}
      \item Fahrplan- Applikation?
      \begin{itemize}
        \item Anzeige des Zugfahrplans gesamt
        \item Anzeige des Fahrplans an bestimmter Haltestelle
        \item \textit{Gab es schon...}
      \end{itemize}
      \item Kinoinformation?
      \begin{itemize}
        \item Anzeige der aktuell laufenden Filme
        \item Anzeige der demn\"achst laufenden Filme
      \end{itemize}
      \item Wetterinformation? \textbf{weatherinfo}
    \end{itemize}
  \end{itemize}
\end{frame}
\begin{frame}
  \frametitle{Anforderungen an die Beispielapplikation}
  \begin{itemize}
    \item Welche Wetterdaten sollen dargestellt werden, f\"ur welchen Zeitraum und f\"ur welche St\"adte? \textbf{Model}
    \item Welche Anzeigearten wollen wir implementieren? \textbf{View}
    \item Welche Funktionalit\"aten in den Views sollen implementiert werden? \textit{Controller}
  \end{itemize}
\end{frame}
\begin{frame}
  \frametitle{Das Model}
  \begin{itemize}
    \item Was ist als Wetterinformation sinnvoll?
    \begin{itemize}
      \item Temperatur
      \item Bew\"olkung
      \item Windst\"arke
      \item Windrichtung
    \end{itemize}
    \item Ein Zeitraum von 5 Tagen (\textit{l\"angere Vorhersagen grenzen an Wahrsagerei})
    \item Welche St\"adte und welche Zusatzinformationen?
    \begin{itemize}
      \item Dresden,Oslo,Springfield,...
      \item Weltkoordinaten (L\"angen- und Breitengrad) f\"ur die Ortsbestimmung
    \end{itemize}
  \end{itemize}
\end{frame}
\begin{frame}
  \frametitle{Die Views}
  \begin{itemize}
    \item Welche Views?
    \begin{itemize}
      \item Verlaufskurve der Temperatur: \textbf{temperature\underline{ }view}
      \item Wetterinformationen f\"ur eine bestimmte Stadt: \textbf{day\underline{ }view}
      \item Anzeige der Temperatur und Bew\"olkung in Tabellenform: \textbf{table\underline{ }view}
      \item Anzeige der Bew\"olkung auf einer Weltkarte: \textbf{world\underline{ }view}
    \end{itemize}
    \item Design der Oberfl\"achen der Views
    \begin{itemize}
      \item Per Handzeichnung im ersten Schritt diskutiert und definiert
      \item Nachfolgend dann von Implementierer durch das Framework realisiert
    \end{itemize}
  \end{itemize}
\end{frame}
\begin{frame}
  \frametitle{Die Funktionalit\"at der Views}
  \begin{itemize}
    \item \textbf{temperature\underline{ }view}
    \begin{itemize}
      \item Auswahl der Stadt
    \end{itemize}
    \item \textbf{day\underline{ }view}
    \begin{itemize}
      \item Auswahl der Stadt
      \item Auswahl des angezeigten Tages
    \end{itemize}
    \item \textbf{table\underline{ }view}
    \begin{itemize}
      \item Einschr\"ankung der angezeigten St\"adte durch einen Filter
      \begin{itemize}
        \item Filter soll case-insensitive sein
      \end{itemize}
      \item \"Anderung der Temperatur f\"ur eine Stadt und einen Tag
      \begin{itemize}
        \item Temperatureintrag soll editierbar werden nach einem Doppelklick
      \end{itemize}
    \end{itemize}
    \item \textbf{world\underline{ }view}
    \begin{itemize}
      \item Auswahl des angezeigten Tages
    \end{itemize}
  \end{itemize}
\end{frame}
