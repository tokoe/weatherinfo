%% the main definitions for the report (KOMA-Script based)
\documentclass[11pt,a4paper,titlepage]{scrreprt}
\usepackage[T1]{fontenc}
\usepackage[utf8]{inputenc}
\usepackage[german,english]{babel}
%\usepackage{layouts}
\usepackage{graphicx}
 \usepackage[round]{natbib}

\begin{document}

\selectlanguage{\german}

\title{{\Huge \bf MVC}\\[0.55em]{\LARGE Das Model-View-Controller-Konzept}}
\author{{\bf SE3 Team}\\Silvio Kunaschk}

\date{Juni 2009}
\maketitle

\begin{abstract}
Für die wahlobligatorische Vorlesung {\bf Software Engineering 3} von
Frau Professor Dr. Hauptmann, muß am Ende des Semesters in einer 
Gruppe von Studenten eine Belegarbeit zu einem gewählten Thema
angefertigt werden. Im SS09 wählte unsere Gruppe das Thema
{\bf Model-View-Controller-Konzept}.\\{\smallskip}

Die Zielstellung ist ein Software-System zu entwickeln zur Simulation
des MVC-Konzeptes. Unsere Gruppe hat sich dazu entschieden, das Konzept
mit Hilfe des {\itshape Model/View Programming Framework} von {\itshape Qt}
zu erläutern.\\{\smallskip}

Dieses Dokument beschreibt den allgemeinen Ansatz des 
Model-View-Controller-Konzeptes und erörtert dazu, anhand der Gesichtspunkte
eines Frameworks, die Umsetzung des Model-View-Controller Paradigma im
Model/View~Programming~Framework.\\{\bigskip}

Die Gruppe besteht aus:
\begin{itemize}
\item Uwe Hausbrandt
\item Tobias König
\item Silvio Kunaschk
\end{itemize}

\end{abstract}

\tableofcontents

\chapter{Allgemeine Betrachtungen}
In der Entwicklung von Software-Systemen haben sich (letztendlich) die Systeme durchgesetzt, 
welche mit Hilfe von graphischen Bedienoberflächen es dem Benutzer erlauben, das System
bzw. eine Anwendung, interaktiv und nutzerfreundlich zu bedienen.\\

Dem Benutzer wird dadurch ein einfacher Zugriff auf das Software-System ermöglicht und
dabei geholfen, eine Anwendung zu verstehen und schneller und komfortabler damit zu arbeiten.

An der Akzeptanz und Verbreitung von Computer-Systemen in nahezu allen Bereichen der Industrie
und Wirtschaft, und in letzter Zeit auch immer mehr im privaten gesellschaftlichen Bereich, und
der (sehr wahrscheinlichen) Annahme, dass die meisten Anwender nur über rudimentäre
Informatikkenntnisse verfügen, aber trotzdem in der Lage sind mit Hilfe der graphischen
Benutzerschnittstelle die verschiedensten Aufgaben mit Software-Systemen einfach und effizient
zu lösen, kann man erkennen, welchen Stellenwert dieser Bestandteil von interaktiven
Software-Systemen besitzt.

Möchte man, dass sich sein Anwendungssystem bzw. Applikation durchsetzt, ist es heutzutage
unabdingbar eine ausgefeilte Benutzerschnittstelle mit zu implementieren, oder aber 
das System so zu entwickeln, dass dies von jeweiligen Fachleuten übernommen werden kann.

Spezifiziert man die Architektur eines interaktiven Software-Systems, muß man den funktionalen
Teil von der Bedienschnittstelle unabhängig halten. So können Änderungen an den Teilen
unabhängig voneinander durchgeführt werden, z.B. um die Benutzerschnittstelle an andere
Bedürfnisse anzupassen, ohne Auswirkungen auf den Funktionalen Teil des Systems.

Für diese grundlegende strukurelle Organisation interaktiver Software-Systeme wendet man
das Architekturmuster Model-View-Controller an.

\chapter{Das Model-View-Controller Muster}
\section{Das MVC Paradigma}
Das Model-View-Controller Muster teilt eine interaktive Anwendung in drei Komponenten auf.

\begin{figure}[h]
\fbox{\includegraphics[width=14cm]{mvc-schema.png}}
\caption{Die MVC Komponenten in Aktion}
\end{figure}

\begin{description}
\item[Model]
Das  Model stellt das Anwendungsobjekt dar. Es enthält die gesamte Daten, Zustands-
und Anwendungslogik. Es weiß nichts über Views und Controller, es bietet allerdings eine
Schnittstelle an, über die sein Zustand beeinflußt und abgerufen werden kann. Außerdem
kann es Benachrichtigungen über seine Zustandsänderungen an seine Beobachter senden.

\item[View]
Die View ist die Bildschirmrepräsentation des Anwendungsobjektes. Sie erhält den Zustand
und die Daten (normalerweise) direkt vom Model.

\item[Controller]
Der Controller bestimmt die Möglichkeiten, mit denen die Benutzungsschnittstelle auf
Benutzereingaben reagieren kann. Er nimmt die Eingaben des Benutzers entgegen und stellt
fest, was diese für das Model bedeuten. Er verarbeitet die Bedieneingaben.
\end{description}

Die View- und die Controllerkomponente beschreiben zusammen die Bedienschnittstelle.
Ein Benachrichtigungsmechanismus sichert die Konsistenz zwischen der Bedienschnittstelle
und dem Model.

Das MVC-Paradigma entkoppelt die Benutzungsschnittstellen der View vom Model. Dies
erhöht die Flexibilität und die Wiederverwendbarkeit.

\section{Das MVC etwas genauer betrachtet}
Schaut man beim Model-View-Controller etwas genauer hin, erkennt man, dass das MVC
ein Satz von Mustern ist, die in einem Entwurf zusammenarbeiten.\\
Es besteht eigentlich aus 3 Entwurfsmustern die hier aber nur kurz erläutert werden
sollen.

\subsection{Observer-Muster}
Das wichtigste Muster beim MVC-Entwurf, auch für dessen Verständis, ist das Observer-Muster,
weshalb es hier jetzt genauer betrachtet wird.

\begin{description}
\item[Def.]
Definiere eine 1-zu-n-Abhängigkeit, zwischen Objekten, so dass die Änderung des
Zustands eines Objektes dazu führt, dass alle abhängigen Objekte benachrichtigt
und automatisch aktualisiert werden.
\end{description}

%% erzwinge das Anzeigen des Bildes genau an dieser Stelle
\enlargethispage{1cm}
\begin{figure}[h]
\fbox{\includegraphics[width=14cm]{observer-schema.png}}
\caption{Das Observer-Muster in Aktion}
\end{figure}

Das Model nutzt das Observer-Muster, um View und Controller auf dem aktuellen
Stand über die letzten Zustandsänderungen zu halten.
Es macht das Model völlig unabhängig von View und Controller. So können für das gleiche
Model unterschiedliche, oder sogar mehrere Views auf einmal verwendet werden.

\begin{figure}[h]
\fbox{\includegraphics[width=14cm]{observer-kd.png}}
\caption{Das Observer-Muster Klassendiagramm}
\end{figure}


\subsection{Strategy-Muster}
\begin{description}
\item[Def.]
Definiere eine Familie von Algorithmen, kapsele jeden einzelnen und mache sie austauschbar.
Das Strategy-Muster ermöglicht es, den Algorithmus unabhängig von ihn nutzenden Klienten
zu variieren.
\end{description}

Die View und der Controller implementieren das Strategy-Muster. Der Controller ist das
Verhalten der View und kann leicht gegen einen anderen Controller ausgetauscht werden,
wenn ein anderes Verhalten gewünscht wird.

Die View ist ein Objekt, das mit einer Strategie konfiguriert ist; der Controller
liefert diese Strategie. Die View ist nur für die Anzeige der Anwendung zuständig;
alle Entscheidungen über das Verhalten der Schnittstelle {\itshape delegiert} sie an
den Controller. Durch die Verwendung des Strategy-Muster bleibt die View vom
Model {\itshape entkoppelt}, denn der Controller ist ja bei der Bearbeitung der
Benutzeraktionen für die Interaktion mit dem Model zuständig. Die View weiß nicht
wie das vor sich geht.

\begin{figure}[h]
\fbox{\includegraphics[width=14cm]{strategy-schema.png}}
\caption{Das Strategy-Muster in Aktion}
\end{figure}

Die View delegiert die Verarbeitung der Benutzeraktionen an den Controller. Der Controller
übersetzt die Eingaben des Benutzers in Aktionen auf dem Model.

\subsection{Composite-Muster}
Die View ist ein Kompositum aus GUI-Komponenten (Labels, Buttons, Texteingabefentern, usw. ).
Die oberste Komponente enthält andere Komponenten, die wiederum weitere Komponenten enthalten,
bis man beim Blattknoten angelangt ist. Sie verwendet dieses Muster intern, um die Bestandteile
der Anzeige zu verwalten.

\section{Nachteile von MVC}

\begin{description}
\item[Größere Komplexität.]
Nicht immer ist die strikte Einhaltung der Model-View-Controller-Struktur die beste Art, eine
interaktive Anwendung zu entwickeln. Es kann sein das die Verwendung vom MVC die Komplexität
der Anwendung erhöht ohne den Zugewinn an Flexibilität.

\item[Potential für eine übermäßige Anzahl von Aktualisierungen.] Hat zum Beispiel eine
einfache Aktion des Anwenders viele Aktualisierungen zur Folge, sollte das Model unnötige
Benachrichtigungen über Änderungen auslassen. Eine View die gerade nicht sichtbar ist, braucht
nicht benachrichtigt zu werden.

\item[Enge Verbindung zwischen View- und Controllerkomponenten.] View und Controller sind
eigene, aber eng gekoppelte Komponenten, was deren jeweilige Wiederverwendung behindert.
Es ist unwahrscheinlich, dass eine View ohne ihre Controllerkomponente oder umgekehrt
verwendet wird.
\end{description}

\chapter{Framework}
\section{Ein kurzer Überblick}
Ein Framework besteht aus einer Menge von zusammenarbeitenden Klassen, die einen
wiederverwendbaren Entwurf für eine bestimmte Klasse von Software darstellen. Das
Framework bestimmt die Architektur der Anwendung. Es definiert:

\begin{itemize}
\item die Struktur im Großen
\item Unterteilung in Klassen und Objekte
\item die jeweiligen zentralen Zuständigkeiten
\item die Zusammenarbeit der Klassen und Objekte sowie den Kontrollfluß
\end{itemize}

Ein Framework legt diese Entwurfsparameter im voraus fest, so dass der Entwickler
sich auf die spezifischen Details seiner Anwendung konzentrieren kann.
Verwendet man ein Framework, schreibt man den Code, der vom Framework gerufen wird.
Dies wird erreicht indem man Operationen mit bestimmten Namen und Aufrufkonventionen
schreiben muß. Dies reduziert die zu treffenden Entwurfsentscheidungen. Diese sind
bereits von anderen getroffen worden.

Speziell bei GUI-Frameworks ermöglicht diese Vorgehensweise ein sehr viel schnelleres
Entwickeln von Anwendungen\footnote{in diesem Fall von den View-Komponenten
der Anwendung}. Diese haben eine ähnliche Struktur und sind einfacher Wiederverwendbar.
Auf der einen Seite schränkt dies die Kreativität ein, aber man erhält Komponenten 
in denen Erfahrung steckt und die schon erprobt sind. Dem Benutzer der Anwendung sind
diese Verhaltensweisen in der Regel schon bekannt.

\chapter{Das Model/View Programming Framework von Qt}
\section{Was ist Qt?}
Qt ist das de facto Standard C++ Framework für ein sehr schnelles Entwickeln von
Cross-Platform-Software. Zusätzlich zu einer sehr Umfangreichen C++ Klassenbibliothek
enthält Qt Werkzeuge, die das Schreiben von Anwendungen erleichtern und
beschleunigen.

Qt enthält eine große Menge von Widgets, welche Standard GUI Funktionalität zur
Verfügung stellen. Qt ist ein weltweit verwendetes und ausgereiftes C++ Framework.

Zu den vielen kommerziellen Verwendungen von Qt, ist die Open Source Edition
von Qt die Grundlage von KDE, der Linux Desktop Umgebung.

\newpage
\nocite{*}
\addcontentsline{toc}{chapter}{Literaturverzeichnis}
\bibliography{mvc-literatur}{}
\bibliographystyle{plain}

\end{document}
